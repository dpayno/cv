%%%%%%%%%%%%%%%%%%%%%%%%%%%%%%%%%%%%%%%%%%%%%%%%
%%% CV Daniel Payno Zarceño 03/2025 %%%%%%%%%%%%
%%%%%%%%%%%%%%%%%%%%%%%%%%%%%%%%%%%%%%%%%%%%%%%%

\documentclass[10pt,a4paper,ragged2e,withhyper]{altacv}

% Change the page layout if you need to
\geometry{left=1.25cm,right=1.25cm,top=1.5cm,bottom=1.5cm,columnsep=1.2cm}

% The paracol package lets you typeset columns of text in parallel
\usepackage{paracol}

% Change the font if you want to, depending on whether
% you're using pdflatex or xelatex/lualatex
% WHEN COMPILING WITH XELATEX PLEASE USE
% xelatex -shell-escape -output-driver="xdvipdfmx -z 0" sample.tex
\iftutex
  % If using xelatex or lualatex:
  \setmainfont{Roboto Slab}
  \setsansfont{Lato}
  \renewcommand{\familydefault}{\sfdefault}
\else
  % If using pdflatex:
  \usepackage[rm]{roboto}
  \usepackage[defaultsans]{lato}
  % \usepackage{sourcesanspro}
  \renewcommand{\familydefault}{\sfdefault}
\fi

\definecolor{SlateGrey}{HTML}{2E2E2E} % Gris oscuro
\definecolor{LightGrey}{HTML}{666666} % Gris medio
\definecolor{DarkPastelRed}{HTML}{1B2B3A} % Azul muy oscuro
\definecolor{PastelRed}{HTML}{2D4A66} % Azul oscuro
\definecolor{GoldenEarth}{HTML}{6C8EA4} % Azul grisáceo claro
\colorlet{name}{black}
\colorlet{tagline}{PastelRed} % Azul oscuro
\colorlet{heading}{DarkPastelRed} % Azul muy oscuro
\colorlet{headingrule}{GoldenEarth} % Azul grisáceo claro
\colorlet{subheading}{PastelRed} % Azul oscuro
\colorlet{accent}{PastelRed} % Azul oscuro
\colorlet{emphasis}{SlateGrey} % Gris oscuro
\colorlet{body}{LightGrey} % Gris medio


% Change some fonts, if necessary
\renewcommand{\namefont}{\Huge\rmfamily\bfseries}
\renewcommand{\personalinfofont}{\footnotesize}
\renewcommand{\cvsectionfont}{\LARGE\rmfamily\bfseries}
\renewcommand{\cvsubsectionfont}{\large\bfseries}


% Change the bullets for itemize and rating marker
% for \cvskill if you want to
\renewcommand{\cvItemMarker}{{\small\textbullet}}
\renewcommand{\cvRatingMarker}{\faCircle}
% ...and the markers for the date/location for \cvevent
% \renewcommand{\cvDateMarker}{\faCalendar*[regular]}
% \renewcommand{\cvLocationMarker}{\faMapMarker*}


% If your CV/résumé is in a language other than English,
% then you probably want to change these so that when you
% copy-paste from the PDF or run pdftotext, the location
% and date marker icons for \cvevent will paste as correct
% translations. For example Spanish:
% \renewcommand{\locationname}{Ubicación}
% \renewcommand{\datename}{Fecha}

\begin{document}
\name{Daniel Payno Zarceño}
\tagline{Telecommunications \& electronics engineer}
%% You can add multiple photos on the left or right
\photoR{2.1cm}{dani}
% \photoL{2.5cm}{Yacht_High,Suitcase_High}

\personalinfo{%
  % Not all of these are required!
  \email{payno97@gmail.com}
  \phone{+34 638 587 538}
  % \mailaddress{Åddrésş, Street, 00000 Cóuntry}
  \location{Madrid, Spain}
  % \homepage{www.homepage.com}
  % \twitter{@twitterhandle}
  % \xtwitter{@x-handle}
  \linkedin{daniel-payno-zarceño}
  \github{dpayno}
  % \orcid{0000-0000-0000-0000}
  %% You can add your own arbitrary detail with
  %% \printinfo{symbol}{detail}[optional hyperlink prefix]
  % \printinfo{\faPaw}{Hey ho!}[https://example.com/]

  %% Or you can declare your own field with
  %% \NewInfoFiled{fieldname}{symbol}[optional hyperlink prefix] and use it:
  % \NewInfoField{gitlab}{\faGitlab}[https://gitlab.com/]
  % \gitlab{your_id}
  %%
  %% For services and platforms like Mastodon where there isn't a
  %% straightforward relation between the user ID/nickname and the hyperlink,
  %% you can use \printinfo directly e.g.
  % \printinfo{\faMastodon}{@username@instace}[https://instance.url/@username]
  %% But if you absolutely want to create new dedicated info fields for
  %% such platforms, then use \NewInfoField* with a star:
  % \NewInfoField*{mastodon}{\faMastodon}
  %% then you can use \mastodon, with TWO arguments where the 2nd argument is
  %% the full hyperlink.
  % \mastodon{@username@instance}{https://instance.url/@username}
}

\makecvheader
%% Depending on your tastes, you may want to make fonts of itemize environments slightly smaller
% \AtBeginEnvironment{itemize}{\small}

%% Set the left/right column width ratio to 6:4.
\columnratio{0.6}

% Start a 2-column paracol. Both the left and right columns will automatically
% break across pages if things get too long.
\begin{paracol}{2}

\cvsection{About me}

Telecommunications engineer with a deep passion for electronics and programming, specializing in embedded systems and Linux-based operating systems. I enjoy embracing new challenges and contributing to innovative work environments.

\cvsection{Experience}

\cvevent{Hardware \& firmware engineer}{TycheTools}{August 2020 -- April 2025}{Madrid, ES}

Company specialized in \textbf{data center energy efficiency}, focused on environmental monitoring, and cooling system control.

\begin{itemize}
    \item \textbf{Hardware engineering}: requirement gathering, PCB design and manufacturing, assembly, prototyping, and testing. Purchasing management, communication with suppliers, and RFQs.
    \item \textbf{Firmware engineering}: design and implementation of embedded systems, including microcontroller programming, communication protocols, and sensor networks.
    \item \textbf{Software development}: development of Linux-based systems using Yocto, and implementation of C and Python applications.
    \item \textbf{Data center experience}: installation of sensor networks, cooling optimization and energy consumption management.
\end{itemize}

\divider

\cvevent{Electronic Research Internship}{GreenLSI (Department of Electronic Engineering, ETSIT)}{May 2020 -- August 2020}{Madrid, ES}
\begin{itemize}
    \item Research and development of Bluetooth Mesh networks based on Nordic nRF52 microcontrollers.
\end{itemize}

\cvsection{Technical skills}

\begin{itemize}
  \item \textbf{Programming languages}:
  
  \smallskip
  \cvtag{C/C++} \cvtag{\simpleicon{python} Python} \cvtag{\simpleicon{gnubash} Bash} \cvtag{\simpleicon{dart} Dart}
  
  \item \textbf{Microcontrollers}:

  \smallskip
  \cvtag{\simpleicon{nordicsemiconductor} Nordic} \cvtag{\simpleicon{espressif} ESP} \cvtag{\simpleicon{stmicroelectronics} ST} \cvtag{AVR} \cvtag{PIC} \cvtag{ARM}

  \item \textbf{Communication protocols}:
  
  \smallskip
  \cvtag{I2C} \cvtag{SPI} \cvtag{UART} \cvtag{RS232} \cvtag{RS485} \cvtag{Wi-Fi} \cvtag{BLE} \cvtag{LoRa} \cvtag{NFC} \cvtag{MQTT} \cvtag{Modbus} \cvtag{SNMP} \cvtag{Raft}

  \item \textbf{Development tools}:
  
  \smallskip
  \cvtag{\simpleicon{git} Git} \cvtag{\simpleicon{linux} Linux} \cvtag{\simpleicon{make} Makefile} \cvtag{\simpleicon{docker} Docker} \cvtag{\simpleicon{yocto} Yocto} \cvtag{\simpleicon{latex} \LaTeX} \cvtag{\simpleicon{markdown} Markdown} \cvtag{\simpleicon{platformio} PlatformIO}

  \item \textbf{Design tools}:

  \smallskip
  \cvtag{KiCad} \cvtag{2D/3D CAD} \cvtag{\simpleicon{blender} Blender}

  \item \textbf{Others}:

  \smallskip
  \cvtag{API REST} \cvtag{\simpleicon{nginx} NGINX} \cvtag{\simpleicon{mysql} SQL} \cvtag{Django} \cvtag{FastAPI} \cvtag{LVGL} \cvtag{\simpleicon{flutter} Flutter} \cvtag{\simpleicon{raspberrypi} Raspberry Pi} \cvtag{\simpleicon{wireshark} Wireshark} \cvtag{\simpleicon{jira} Jira} \cvtag{Agile}

\end{itemize}

%% Switch to the right column. This will now automatically move to the second
%% page if the content is too long.
\switchcolumn

\cvsection{Education}

\cvevent{Master of Science in Electronic \\Systems Engineering}{Universidad Politécica de Madrid, ETSIT}{2021 -- 2022}{Madrid, ES}

\faGraduationCap Final Master's Degree Project: Design and implementation of a sensor network for air quality monitoring (10/10)

\faTrophy Award for the best record in the 2021-2022 academic year. Sponsored by Fundetel.

\divider

\cvevent{Bachelor of Engineering in \\Telecommunication Technologies and Services Engineering}{Universidad Politécica de Madrid, ETSIT}{2015 -- 2020}{Madrid, ES}

\faGraduationCap Final Degree Project: Design and implementation of a MIDI controller for aerophone instruments (10/10)

\cvsection{Languages}

\cvskill{Spanish}{5}
\divider
\cvskill{English}{4.5}

\cvsection{Softskills}

\begin{itemize}
  \setlength{\itemindent}{0.5em}
  \item \textbf{Problem-solving}: experience in project development with multi-disciplinary teams.
  \item \textbf{Teamwork}: ability to analyze complex situations and find effective solutions.
  \item \textbf{Adaptability, flexibility}: willingness to learn and adapt to new technologies and work environments.
  \item \textbf{Agile methodologies, Scrum}: experience in project management using agile methodologies.
\end{itemize}

\medskip

\cvsection{Others}

\begin{itemize}
  \setlength{\itemindent}{0.5em}
  \item \textbf{DIY} projects: 3D printing, \textbf{self-hosted} services
  \cvtag{\simpleicon{immich} Immich} \cvtag{\simpleicon{influxdb}InfluxDB}\\ \cvtag{\simpleicon{homeassistant} Home Assistant}\cvtag{\simpleicon{grafana} Grafana}
  \item Design of electronic music devices.
  \item \textbf{Driving license} B and own car.
\end{itemize}

% \cvsection{Strengths}

% % Don't overuse these \cvtag boxes — they're just eye-candies and not essential. If something doesn't fit on a single line, it probably works better as part of an itemized list (probably inlined itemized list), or just as a comma-separated list of strengths.

% % The `ragged2e` document class option might cause automatic linebreaks between \cvtag to fail.
% % Either remove the ragged2e option; or 
% % add \LaTeXraggedright in the paragraph for these \cvtag
% {\LaTeXraggedright
% \cvtag{Hard-working}
% \cvtag{Eye for detail}
% \cvtag{Motivator \& Leader}
% \par}

% \divider\smallskip

% %% ...Or manually add linebreaks yourself
% \cvtag{C}
% \cvtag{Embedded Systems}\\
% \cvtag{Statistical Analysis}

%% Yeah I didn't spend too much time making all the
%% spacing consistent... sorry. Use \smallskip, \medskip,
%% \bigskip, \vspace etc to make adjustments.

\end{paracol}


\end{document}
